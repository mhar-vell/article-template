%<<<<<<<<<<<<<<<<<<<<<<<<<<<<<<<<<<<<<<<<<<<<<<<<<<<<<<<<<<<<<<<<<<<<<<<<<<<<<-
% Template for Articles -------------------------------------------------------
% 2021 version ----------------------------------------------------------------
%
\documentclass{article}
%
%<<<<<<<<<<<<<<<<<<<<<<<<<<<<<<<<<<<<<<<<<<<<<<<<<<<<<<<<<<<<<<<<<<<<<<<<<<<<<-
% Sample Document LaTeX packages ----------------------------------------------
%
\usepackage[utf8]{inputenc}
\usepackage{article}
\usepackage{amsmath}
\usepackage[hidelinks]{hyperref}
\usepackage{url}
\usepackage{graphicx}
\usepackage{booktabs}
\usepackage{lipsum}
%
\usepackage[brazil, english]{babel} % supports the terms in English and Portuguese
\usepackage[T1]{fontenc}
%
%<<<<<<<<<<<<<<<<<<<<<<<<<<<<<<<<<<<<<<<<<<<<<<<<<<<<<<<<<<<<<<<<<<<<<<<<<<<<<-
% Declare the path(s) where your graphic files are ----------------------------
\graphicspath{{figs/}}
\DeclareGraphicsExtensions{.pdf,.jpeg,.png,.jpg}
%
%<<<<<<<<<<<<<<<<<<<<<<<<<<<<<<<<<<<<<<<<<<<<<<<<<<<<<<<<<<<<<<<<<<<<<<<<<<<<<-
% Title and Author information ------------------------------------------------
%
\title{Title Article Template for SIINTEC}
%
\author{%
    First Author1\thanks{Senai Cimatec}, %
    Second Author2\samethanks , %
    Third Author3\thanks{Research Institute Two, Campus Rd. 4, 12345 Flatland} and %
    %and %
    Fourth Author4\samethanks%
\protect\footnotemark[2]}
%
\date{}
%
%<<<<<<<<<<<<<<<<<<<<<<<<<<<<<<<<<<<<<<<<<<<<<<<<<<<<<<<<<<<<<<<<<<<<<<<<<<<<<-
% Additional Paper Information ------------------------------------------------
% Article Type - Uncomment and modify, if necessary. --------------------------
% Accepted values: research, overview, and dataset ----------------------------
% \type{research}
%
% Citation in First Page ------------------------------------------------------
% "Mandatory" (if missing will print the complete list of authors, ------------
% including the \thanks symbols) ----------------------------------------------
\authorref{Author1,~F., Author2,~S., Author3,~T. and Author4,~F.}
%
% (Optional)
% \journalyear{2017}
% \journalvolume{V}
% \journalissue{N}
% \journalpages{xx--xx}
% \doi{xx.xxxx/xxxx.xx}
%
% Remaining Pages (Optional) --------------------------------------------------
\authorshort{Author1, Author2, Author3 and Author4} %or, e.g., 
%\authorshort{Author1 et al}
% \titleshort{Template for SIINTEC}
%<<<<<<<<<<<<<<<<<<<<<<<<<<<<<<<<<<<<<<<<<<<<<<<<<<<<<<<<<<<<<<<<<<<<<<<<<<<<<-
% Document Content ------------------------------------------------------------
%
\begin{document}
%
%<<<<<<<<<<<<<<<<<<<<<<<<<<<<<<<<<<<<<<<<<<<<<<<<<<<<<<<<<<<<<<<<<<<<<<<<<<<<<-
% Abstract --------------------------------------------------------------------
%
\twocolumn[{%
%
\maketitleblock
%
\selectlanguage{english}
\begin{abstract}
  Research articles must have the main text prefaced by an abstract of no more than 250 words summarising the main arguments (objetive, methodology, main results) and conclusions of the article. It should not exceed 10 lines.
  This must have the heading "Abstract" and be easily identified from the start of the main text.
\end{abstract}
%
\begin{keywords}
  must include 3 to 5 keywords, separated by semicolons.
\end{keywords}
%
\selectlanguage{brazil}
\begin{abstract}
  Os artigos de pesquisa devem ter o texto principal precedido por um resumo de no máximo 250 palavras resumindo os principais argumentos (objetivo, metodologia, principais resultados) e as conclusões do artigo. Não deve exceder 10 linhas.
  Deve ter o título "Resumo" e ser facilmente identificado desde o início do texto principal.
\end{abstract}
%
\begin{keywords}
  deve ter no mínimo de 3 a 5 palavras-chave, em português, separadas por ponto e vírgula.
\end{keywords}
}]
%
\saythanks{}
%
%<<<<<<<<<<<<<<<<<<<<<<<<<<<<<<<<<<<<<<<<<<<<<<<<<<<<<<<<<<<<<<<<<<<<<<<<<<<<<-
% Main Content Start ----------------------------------------------------------
\section{Headings}\label{sec:headings}

Up to three level headings may be present and must be clearly identifiable
using different font sizes, bold or italics. IF accepted for publication,
these will be converted into journal during typesetting.

As a general rule, submissions should be structured so that introduction,
methods, conclusions and discussion are all clearly indicated by a first level heading.

\subsection{Level heading 2}

\lipsum[66]

\subsubsection{Level heading 3}

\lipsum[75]

\section{Footnotes}\label{sec:footnotes}

Use endnotes rather than footnotes
(we refer to these as ``Notes'' in the online publication).
These will appear at the end of the main text, before ``References''.
All notes should be used only where crucial clarifying information
needs to be conveyed.
Avoid using notes for purposes of referencing, with in-text citations used
instead.
If in-text citations cannot be used, a source can be cited as part of a note.
Please insert the endnote marker after the end punctuation.

\section{Figures}\label{sec:figures}

Figures must all be cited in the main text, in sequential order.
All figures should be placed within the text file upon submission and during
the review process. Figures/images have a resolution of at least 150 dpi
(300 dpi or above preferred). The files are in one of the following formats:
JPG, TIFF, GIF, PNG, EPS (to maximise quality,
the original source file is preferred).

After editorial acceptance, you will be asked to provide the figure
images in high resolution files, rather than in the submission file,
so that this quality is transferred into the publication.
The typesetting process will place the figures in an appropriate
location in the PDF.

Each figure must have an accompanying descriptive main title.
This should clearly and concisely summarise the content and/or
use of the figure image.
A short additional figure legend is optional to offer a further description.

\begin{figure}[htbp]
  \centering
  \includegraphics[width=0.85\columnwidth]{figure.png}
  \caption{Figure captions should be placed below the figure.}
\label{fig:figure}
\end{figure}

\section{Tables}\label{sec:tables}

Tables must be created using a word processor's table function, not tabbed text.
Tables should be included in the manuscript.
The final layout will place the tables as close to their first citation as possible.

All tables must be cited within the main text, numbered with Arabic numerals in consecutive order (e.g. Table 1, Table 2, etc.).

Each table must have an accompanying descriptive title.
This should clearly and concisely summarise the content and/or use of the table.
A short additional table legend is optional to offer a further description of the table.

\subsection{Tables should not include}

\begin{itemize}
  \item Rotated text
  \item Colour to denote meaning (it will not display the same on all devices)
  \item Images
  \item Vertical or diagonal lines
  \item Multiple parts (e.g. ``Table 1a'' and ``Table 1b'').
  These should either be merged into one table,
  or separated into ``Table 1'' and ``Table 2''.
\end{itemize}

\begin{table}[htpb]
\centering
  \begin{tabular}{ll}
  \toprule
  \bfseries String Value & \bfseries Numeric Value \\ \midrule
  Hello ISMIR  & 2017          \\
  \bottomrule
  \end{tabular}
  \caption{Table captions should be placed below the table.}
\label{tab:table}
\end{table}

\section{Equations}\label{sec:equations}

Equations should be placed on separate lines and numbered.
The number should be on the right side, in parentheses,
as in Eqn.~(\ref{eq:eq}).

\begin{align}\label{eq:eq}
E = mc^2
\end{align}

\section{Reproducibility (if applicable)}

If the content of your submission relates to data or software
that has been deposited in a code or preservation repository,
please provide summary information here, along with a DOI that
links to the deposited code/data.

\section{Competing interests}

If any of the authors have any competing interests then these
must be declared. A short paragraph should be placed before
the references.
Guidelines for competing interests are available online.%
\endnote{Link to guidelines: \\
  \url{http://www.senaicimatec.com.br/en/}}

\section{Ethics and consent}

Research involving human subjects, human material, or human data,
must have been performed in accordance with the Declaration of Helsinki.
Where applicable, the studies must have been approved by an appropriate
ethics committee and the authors should include a statement within
the article text detailing this approval, including the name of the ethics committee and reference number of the approval.
The identity of the research subject should be anonymised whenever possible.
For research involving human subjects, informed consent to participate
in the study must be obtained from participants (or their legal guardian).
-


\section{References}

All citations must be listed at the end of the text file,
in alphabetical order of authors' surnames.
References should not be listed if they are not cited in
the main text.

In this template, you can use \verb=\citep{}= to include references surrounded by parentheses, such as~\citep{KneesS16_MusicSimilarityRetrieval_SPRINGER}, and \verb=\cite{}=
for references embedded in the text,
such as~\cite{WeihsJVR16_MusicDataAnalysis_CRC},
\cite{SerraEtAl13_RoadmapMIR_CreativeCommon},
\cite{Lerch15_AudioContentAnalysis_WILEY},
or~\cite{Mueller15_FMP_SPRINGER}.
%
%<<<<<<<<<<<<<<<<<<<<<<<<<<<<<<<<<<<<<<<<<<<<<<<<<<<<<<<<<<<<<<<<<<<<<<<<<<<<<-
% Please do not touch. --------------------------------------------------------
% Print Endnotes --------------------------------------------------------------
\IfFileExists{\jobname.ent}{
   \theendnotes
}{
   %no endnotes
}
%
%<<<<<<<<<<<<<<<<<<<<<<<<<<<<<<<<<<<<<<<<<<<<<<<<<<<<<<<<<<<<<<<<<<<<<<<<<<<<<-
% Acknowledgements ------------------------------------------------------------
\section*{Acknowledgements}

Any acknowledgements must be headed and in a separate paragraph, placed after the main text but before the reference list.
%
%<<<<<<<<<<<<<<<<<<<<<<<<<<<<<<<<<<<<<<<<<<<<<<<<<<<<<<<<<<<<<<<<<<<<<<<<<<<<<-
% Bibliography ----------------------------------------------------------------
%
% For bibtex users:
\bibliography{article-template}
%
% For non bibtex users:
%\begin{thebibliography}{citations}
%
%\bibitem {Author:00}
%E. Author.
%``The Title of the Conference Paper,''
%{\it Proceedings of the International Symposium
%on Music Information Retrieval}, pp.~000--111, 2000.
%
%\bibitem{Someone:10}
%A. Someone, B. Someone, and C. Someone.
%``The Title of the Journal Paper,''
%{\it Journal of New Music Research},
%Vol.~A, No.~B, pp.~111--222, 2010.
%
%\bibitem{Someone:04} X. Someone and Y. Someone. {\it Title of the Book},
%    Editorial Acme, Porto, 2012.
%
%\end{thebibliography}
%
%<<<<<<<<<<<<<<<<<<<<<<<<<<<<<<<<<<<<<<<<<<<<<<<<<<<<<<<<<<<<<<<<<<<<<<<<<<<<<-
% The END ---------------------------------------------------------------------
\end{document}